\documentclass{article}
\usepackage{import}
\usepackage{amsmath}
\usepackage{tabularray}
\usepackage{float}
\usepackage{listings}
\usepackage{hyperref}
\usepackage{svg}

\import{lib/latex/}{wgmlgz_course}
\patchcmd{\thebibliography}{\section*}{\section}{}{}

\begin{document}
\itmo[
      variant=Пасека,
      labn=2,
      discipline=Информационные системы,
      group=P3312,
      student=Соколов Анатолий Владимирович \\ Пархоменко Кирилл Александрович,
      teacher=Бострикова Дарья Константиновна
]
\lstset{language=java}
\newgeometry{
  a4paper,
  top=20mm,
  right=10mm,
  bottom=20mm,
  left=30mm
}
\tableofcontents

\section{Отчет второй части}

\textbf{Предметная область:} гео-распределенная пасека

\subsection{Подробное текстовое описание предметной области}

\subsubsection{Улей}
\begin{itemize}
    \item \textbf{Характеристики}: Номер улья, тип улья (например, лежак, многокорпусный), дата установки.
    \item \textbf{Источник}: Основы пчеловодства, где рассматриваются различные типы ульев и их использование [8].
\end{itemize}

\subsubsection{Пчелосемья}
\begin{itemize}
    \item \textbf{Характеристики}: Номер семьи, количество пчел, состояние (здоровая, больная).
    \item \textbf{Источник}: Статья о контроле летной активности пчел и их состоянии [7].
\end{itemize}

\subsubsection{Датчик температуры и влажности}
\begin{itemize}
    \item \textbf{Характеристики}: Идентификатор датчика, значения температуры и влажности, дата и время измерения.
    \item \textbf{Источник}: Описание систем мониторинга в пчеловодстве, где используются датчики для контроля условий в улье [7].
\end{itemize}

\subsubsection{Запись о медосборе}
\begin{itemize}
    \item \textbf{Характеристики}: Дата сбора меда, количество собранного меда, качество.
    \item \textbf{Источник}: Основы пчеловодства и практики сбора меда [8].
\end{itemize}

\subsubsection{Журнал наблюдений}
\begin{itemize}
    \item \textbf{Характеристики}: Дата записи, описание наблюдений (поведение пчел, состояние улья), рекомендации.
    \item \textbf{Источник}: Методические рекомендации по ведению журнала наблюдений за пчелами [8].
\end{itemize}

\subsubsection{Ветеринарный паспорт}
\begin{itemize}
    \item \textbf{Характеристики}: Номер паспорта, дата выдачи, состояние здоровья пчелосемьи.
    \item \textbf{Источник}: Ветеринарные документы для учета здоровья животных на пасеке [8].
\end{itemize}

\subsubsection{Система управления}
\begin{itemize}
    \item \textbf{Характеристики}: Название системы, версия программного обеспечения, дата установки.
    \item \textbf{Источник}: Описание программных решений для управления пасеками [7].
\end{itemize}

\subsubsection{План обслуживания}
\begin{itemize}
    \item \textbf{Характеристики}: Дата планового обслуживания, виды работ (например, осмотр ульев), ответственный за выполнение.
    \item \textbf{Источник}: Рекомендации по техническому обслуживанию ульев и оборудования [7].
\end{itemize}

\subsubsection{Инциденты}
\begin{itemize}
    \item \textbf{Характеристики}: Дата инцидента, описание (например, болезни пчел), принятые меры.
    \item \textbf{Источник}: Нормативные акты по регистрации инцидентов на пасеке [8].
\end{itemize}

\subsubsection{Отчетность по производству}
\begin{itemize}
    \item \textbf{Характеристики}: Период отчета, количество произведенного меда, расходы на содержание пасеки.
    \item \textbf{Источник}: Статья о ведении отчетности в пчеловодстве [8].
\end{itemize}

\begin{itemize}
    \item \textbf{Характеристики}: Период отчета, количество произведенного меда, расходы на содержание пасеки.
    \item \textbf{Источник}: Статья о ведении отчетности в пчеловодстве [8].
\end{itemize}

\subsection{Сформировать ER-модель базы данных (на основе описаний предметной области и прецедентов из предыдущего этапа).}

\begin{figure}[H]
    \centering
    \includegraphics[width=0.8\linewidth]{mermaid-er.png}
    \caption{ER диаграмма}
    \label{fig:enter-label}
\end{figure}

\subsection{Реализовать даталогическую модель в реляционной СУБД PostgreSQL}

\begin{lstlisting}
CREATE OR REPLACE FUNCTION add_constraint_if_not_exists(
    p_table_name TEXT,
    p_constraint_name TEXT,
    p_constraint_sql TEXT
) RETURNS VOID AS $$
BEGIN
    EXECUTE format('ALTER TABLE %I ADD CONSTRAINT %I %s', p_table_name, p_constraint_name, p_constraint_sql);
EXCEPTION
    WHEN duplicate_table THEN
        RAISE NOTICE 'Table constraint %.% already exists', p_table_name, p_constraint_name;
    WHEN duplicate_object THEN
        RAISE NOTICE 'Table constraint %.% already exists', p_table_name, p_constraint_name;
END;
$$ LANGUAGE plpgsql;

DO $$
BEGIN
    PERFORM add_constraint_if_not_exists('apiary', 'check_establishment_date', 'CHECK ("establishment_date" <= CURRENT_DATE)');
    PERFORM add_constraint_if_not_exists('hive', 'check_installation_date', 'CHECK ("installation_date" <= CURRENT_DATE)');
    PERFORM add_constraint_if_not_exists('bee_community', 'check_queen_age', 'CHECK ("queen_age" >= 0)');
    PERFORM add_constraint_if_not_exists('bee_community', 'check_population_estimate', 'CHECK ("population_estimate" >= 0)');
    PERFORM add_constraint_if_not_exists('veterinary_passport', 'check_issue_date', 'CHECK ("issue_date" <= CURRENT_DATE)');
    PERFORM add_constraint_if_not_exists('veterinary_passport', 'check_last_inspection_date', 'CHECK ("last_inspection_date" <= CURRENT_DATE)');
    PERFORM add_constraint_if_not_exists('veterinary_record', 'check_record_date', 'CHECK ("record_date" <= CURRENT_DATE)');
    PERFORM add_constraint_if_not_exists('honey_harvest', 'check_harvest_date', 'CHECK ("harvest_date" <= CURRENT_DATE)');
    PERFORM add_constraint_if_not_exists('honey_harvest', 'check_quantity', 'CHECK ("quantity" >= 0)');
    PERFORM add_constraint_if_not_exists('observation_log', 'check_observation_date', 'CHECK ("observation_date" <= CURRENT_DATE)');
    PERFORM add_constraint_if_not_exists('maintenance_plan', 'check_planned_date', 'CHECK ("planned_date" >= CURRENT_DATE)');
    PERFORM add_constraint_if_not_exists('incident', 'check_incident_date', 'CHECK ("incident_date" <= CURRENT_DATE)');
    PERFORM add_constraint_if_not_exists('production_report', 'check_date_range', 'CHECK ("start_date" <= "end_date")');
    PERFORM add_constraint_if_not_exists('production_report', 'check_total_honey_produced', 'CHECK ("total_honey_produced" >= 0)');
    PERFORM add_constraint_if_not_exists('weather_data', 'check_weather_date', 'CHECK ("date" <= CURRENT_DATE)');
    PERFORM add_constraint_if_not_exists('bee_community', 'unique_hive_id', 'UNIQUE ("hive_id")');
END $$;


DO $$ BEGIN IF NOT EXISTS (
	SELECT
		1
	FROM
		pg_type typ
		INNER JOIN pg_namespace nsp ON nsp.oid = typ.typnamespace
	WHERE
		nsp.nspname = current_schema()
		AND typ.typname = 'role'
) THEN CREATE TYPE role AS ENUM ('ADMIN', 'WORKER');

END IF;

END;

$$ LANGUAGE plpgsql;

CREATE TABLE IF NOT EXISTS "region" (
	"region_id" SERIAL,
	"name" VARCHAR NOT NULL,
	"climate_zone" VARCHAR,
	PRIMARY KEY("region_id")
);

CREATE TABLE IF NOT EXISTS "apiary" (
	"apiary_id" SERIAL,
	"location" VARCHAR NOT NULL,
	"manager_id" INTEGER,
	"establishment_date" DATE,
	PRIMARY KEY("apiary_id")
);

CREATE TABLE IF NOT EXISTS "hive" (
	"hive_id" SERIAL,
	"apiary_id" INTEGER,
	"hive_type" VARCHAR,
	"installation_date" DATE,
	"current_status" VARCHAR,
	PRIMARY KEY("hive_id")
);

CREATE TABLE IF NOT EXISTS "bee_community" (
	"community_id" SERIAL,
	"hive_id" INTEGER,
	"queen_age" INTEGER,
	"population_estimate" INTEGER,
	"health_status" VARCHAR,
	PRIMARY KEY("community_id")
);

CREATE TABLE IF NOT EXISTS "veterinary_passport" (
	"passport_id" SERIAL,
	"bee_community_id" INTEGER,
	"issue_date" DATE,
	"health_status" VARCHAR,
	"last_inspection_date" DATE,
	PRIMARY KEY("passport_id")
);

CREATE TABLE IF NOT EXISTS "veterinary_record" (
	"record_id" SERIAL,
	"passport_id" INTEGER,
	"record_date" DATE,
	"description" TEXT,
	"treatment" TEXT,
	PRIMARY KEY("record_id")
);

CREATE TABLE IF NOT EXISTS "sensor" (
	"sensor_id" SERIAL,
	"hive_id" INTEGER,
	"sensor_type" VARCHAR,
	"last_reading" BYTEA,
	"last_reading_time" TIMESTAMP,
	PRIMARY KEY("sensor_id")
);

CREATE TABLE IF NOT EXISTS "sensor_reading" (
	"reading_id" SERIAL,
	"sensor_id" INTEGER,
	"value" BYTEA,
	"timestamp" TIMESTAMP,
	PRIMARY KEY("reading_id")
);

CREATE TABLE IF NOT EXISTS "honey_harvest" (
	"harvest_id" SERIAL,
	"hive_id" INTEGER,
	"harvest_date" DATE,
	"quantity" FLOAT,
	"quality_grade" VARCHAR,
	PRIMARY KEY("harvest_id")
);

CREATE TABLE IF NOT EXISTS "observation_log" (
	"log_id" SERIAL,
	"hive_id" INTEGER,
	"observation_date" DATE,
	"description" TEXT,
	"recommendations" TEXT,
	PRIMARY KEY("log_id")
);

CREATE TABLE IF NOT EXISTS "maintenance_plan" (
	"plan_id" SERIAL,
	"apiary_id" INTEGER,
	"planned_date" DATE,
	"work_type" VARCHAR,
	"assigned_to" INTEGER,
	PRIMARY KEY("plan_id")
);

CREATE TABLE IF NOT EXISTS "incident" (
	"incident_id" SERIAL,
	"hive_id" INTEGER,
	"incident_date" DATE,
	"description" TEXT,
	"severity" VARCHAR,
	"actions_taken" TEXT,
	PRIMARY KEY("incident_id")
);

CREATE TABLE IF NOT EXISTS "production_report" (
	"report_id" SERIAL,
	"apiary_id" INTEGER,
	"start_date" DATE,
	"end_date" DATE,
	"total_honey_produced" FLOAT,
	"total_expenses" FLOAT,
	PRIMARY KEY("report_id"),
	UNIQUE("apiary_id", "start_date", "end_date")
);

CREATE TABLE IF NOT EXISTS "weather_data" (
	"weather_id" SERIAL,
	"region_id" INTEGER,
	"date" DATE,
	"temperature" FLOAT,
	"humidity" FLOAT,
	"wind_speed" FLOAT,
	"precipitation" FLOAT,
	PRIMARY KEY("weather_id")
);

CREATE TABLE IF NOT EXISTS "user" (
	"user_id" SERIAL,
	"username" VARCHAR NOT NULL UNIQUE,
	"full_name" VARCHAR NOT NULL,
	"role" ROLE,
	"email" VARCHAR NOT NULL UNIQUE,
	"password" VARCHAR NOT NULL,
	"last_login" TIMESTAMP,
	PRIMARY KEY("user_id")
);

CREATE TABLE IF NOT EXISTS "allowed_region" (
	"id" SERIAL NOT NULL,
	"user_id" INTEGER,
	"region_id" INTEGER,
	PRIMARY KEY("id"),
	UNIQUE ("user_id", "region_id")
);

CREATE TABLE IF NOT EXISTS "region_apiary" (
	"id" SERIAL NOT NULL UNIQUE,
	"apiary_id" INTEGER,
	"region_id" INTEGER,
	PRIMARY KEY("id")
);

ALTER TABLE
	"hive"
ADD
	FOREIGN KEY("apiary_id") REFERENCES "apiary"("apiary_id") ON UPDATE NO ACTION ON DELETE CASCADE;

ALTER TABLE
	"bee_community"
ADD
	FOREIGN KEY("hive_id") REFERENCES "hive"("hive_id") ON UPDATE NO ACTION ON DELETE CASCADE;

ALTER TABLE
	"veterinary_passport"
ADD
	FOREIGN KEY("bee_community_id") REFERENCES "bee_community"("community_id") ON UPDATE NO ACTION ON DELETE CASCADE;

ALTER TABLE
	"veterinary_record"
ADD
	FOREIGN KEY("passport_id") REFERENCES "veterinary_passport"("passport_id") ON UPDATE NO ACTION ON DELETE CASCADE;

ALTER TABLE
	"sensor"
ADD
	FOREIGN KEY("hive_id") REFERENCES "hive"("hive_id") ON UPDATE NO ACTION ON DELETE CASCADE;

ALTER TABLE
	"sensor_reading"
ADD
	FOREIGN KEY("sensor_id") REFERENCES "sensor"("sensor_id") ON UPDATE NO ACTION ON DELETE CASCADE;

ALTER TABLE
	"honey_harvest"
ADD
	FOREIGN KEY("hive_id") REFERENCES "hive"("hive_id") ON UPDATE NO ACTION ON DELETE CASCADE;

ALTER TABLE
	"observation_log"
ADD
	FOREIGN KEY("hive_id") REFERENCES "hive"("hive_id") ON UPDATE NO ACTION ON DELETE CASCADE;

ALTER TABLE
	"maintenance_plan"
ADD
	FOREIGN KEY("apiary_id") REFERENCES "apiary"("apiary_id") ON UPDATE NO ACTION ON DELETE CASCADE;

ALTER TABLE
	"incident"
ADD
	FOREIGN KEY("hive_id") REFERENCES "hive"("hive_id") ON UPDATE NO ACTION ON DELETE CASCADE;

ALTER TABLE
	"production_report"
ADD
	FOREIGN KEY("apiary_id") REFERENCES "apiary"("apiary_id") ON UPDATE NO ACTION ON DELETE CASCADE;

ALTER TABLE
	"weather_data"
ADD
	FOREIGN KEY("region_id") REFERENCES "region"("region_id") ON UPDATE NO ACTION ON DELETE CASCADE;

ALTER TABLE
	"allowed_region"
ADD
	FOREIGN KEY("region_id") REFERENCES "region"("region_id") ON UPDATE NO ACTION ON DELETE NO ACTION;

ALTER TABLE
	"region_apiary"
ADD
	FOREIGN KEY("apiary_id") REFERENCES "apiary"("apiary_id") ON UPDATE NO ACTION ON DELETE NO ACTION;

ALTER TABLE
	"region_apiary"
ADD
	FOREIGN KEY("region_id") REFERENCES "region"("region_id") ON UPDATE NO ACTION ON DELETE NO ACTION;

ALTER TABLE
	"allowed_region"
ADD
	FOREIGN KEY("user_id") REFERENCES "user"("user_id") ON UPDATE NO ACTION ON DELETE CASCADE;

ALTER TABLE
	"sensor"
ALTER COLUMN
	"last_reading_time"
SET
	DEFAULT now();

ALTER TABLE
	"sensor_reading"
ALTER COLUMN
	"timestamp"
SET
	DEFAULT now();
\end{lstlisting}

\subsection{Обеспечить целостность данных при помощи средств языка DDL и триггеров}
\begin{lstlisting}
CREATE OR REPLACE FUNCTION add_constraint_if_not_exists(
    p_table_name TEXT,
    p_constraint_name TEXT,
    p_constraint_sql TEXT
) RETURNS VOID AS $$
BEGIN
    EXECUTE format('ALTER TABLE %I ADD CONSTRAINT %I %s', p_table_name, p_constraint_name, p_constraint_sql);
EXCEPTION
    WHEN duplicate_table THEN
        RAISE NOTICE 'Table constraint %.% already exists', p_table_name, p_constraint_name;
    WHEN duplicate_object THEN
        RAISE NOTICE 'Table constraint %.% already exists', p_table_name, p_constraint_name;
END;
$$ LANGUAGE plpgsql;

DO $$
BEGIN
    PERFORM add_constraint_if_not_exists('apiary', 'check_establishment_date', 'CHECK ("establishment_date" <= CURRENT_DATE)');
    PERFORM add_constraint_if_not_exists('hive', 'check_installation_date', 'CHECK ("installation_date" <= CURRENT_DATE)');
    PERFORM add_constraint_if_not_exists('bee_community', 'check_queen_age', 'CHECK ("queen_age" >= 0)');
    PERFORM add_constraint_if_not_exists('bee_community', 'check_population_estimate', 'CHECK ("population_estimate" >= 0)');
    PERFORM add_constraint_if_not_exists('veterinary_passport', 'check_issue_date', 'CHECK ("issue_date" <= CURRENT_DATE)');
    PERFORM add_constraint_if_not_exists('veterinary_passport', 'check_last_inspection_date', 'CHECK ("last_inspection_date" <= CURRENT_DATE)');
    PERFORM add_constraint_if_not_exists('veterinary_record', 'check_record_date', 'CHECK ("record_date" <= CURRENT_DATE)');
    PERFORM add_constraint_if_not_exists('honey_harvest', 'check_harvest_date', 'CHECK ("harvest_date" <= CURRENT_DATE)');
    PERFORM add_constraint_if_not_exists('honey_harvest', 'check_quantity', 'CHECK ("quantity" >= 0)');
    PERFORM add_constraint_if_not_exists('observation_log', 'check_observation_date', 'CHECK ("observation_date" <= CURRENT_DATE)');
    PERFORM add_constraint_if_not_exists('maintenance_plan', 'check_planned_date', 'CHECK ("planned_date" >= CURRENT_DATE)');
    PERFORM add_constraint_if_not_exists('incident', 'check_incident_date', 'CHECK ("incident_date" <= CURRENT_DATE)');
    PERFORM add_constraint_if_not_exists('production_report', 'check_date_range', 'CHECK ("start_date" <= "end_date")');
    PERFORM add_constraint_if_not_exists('production_report', 'check_total_honey_produced', 'CHECK ("total_honey_produced" >= 0)');
    PERFORM add_constraint_if_not_exists('weather_data', 'check_weather_date', 'CHECK ("date" <= CURRENT_DATE)');
    PERFORM add_constraint_if_not_exists('bee_community', 'unique_hive_id', 'UNIQUE ("hive_id")');
END $$;

\end{lstlisting}

\subsection{Реализовать скрипты для создания, удаления базы данных, заполнения базы тестовыми данными.}

\begin{lstlisting}

-- Function to populate the database with test data
CREATE OR REPLACE FUNCTION populate_test_data()
RETURNS VOID AS $$
BEGIN
    -- Insert test data for Region
    INSERT INTO Region (name, climate_zone) VALUES
    ('North', 'Temperate'),
    ('South', 'Mediterranean'),
    ('East', 'Continental'),
    ('West', 'Oceanic');

    -- Insert test data for Apiary
    INSERT INTO Apiary (location, establishment_date, region_id) VALUES
    ('Forest Edge', '2020-05-15', 1),
    ('Meadow', '2019-07-20', 2),
    ('Mountain Side', '2021-03-10', 3),
    ('Coastal Area', '2018-09-01', 4);

    -- Insert test data for Hive
    INSERT INTO Hive (apiary_id, installation_date) VALUES
    (1, '2020-06-01'),
    (1, '2020-06-02'),
    (2, '2019-08-01'),
    (3, '2021-04-01'),
    (4, '2018-10-01');

    -- Insert test data for BeeCommunity
    INSERT INTO BeeCommunity (hive_id, queen_age, population_estimate) VALUES
    (1, 2, 50000),
    (2, 1, 40000),
    (3, 3, 60000),
    (4, 1, 45000),
    (5, 2, 55000);

    -- Insert test data for User
    INSERT INTO "User" (username, password_hash, role) VALUES
    ('admin', 'hashed_password', 'admin'),
    ('user1', 'hashed_password', 'user'),
    ('user2', 'hashed_password', 'user');

    -- Insert test data for UserRegionAccess
    INSERT INTO UserRegionAccess (user_id, region_id) VALUES
    (2, 1),
    (2, 2),
    (3, 3),
    (3, 4);

    -- Insert test data for HoneyHarvest
    INSERT INTO HoneyHarvest (hive_id, harvest_date, quantity) VALUES
    (1, '2021-08-15', 25.5),
    (2, '2021-08-16', 22.0),
    (3, '2021-08-20', 30.5),
    (4, '2021-09-01', 28.0),
    (5, '2021-09-05', 26.5);

    -- Insert test data for ObservationLog
    INSERT INTO ObservationLog (hive_id, user_id, observation_date, notes) VALUES
    (1, 2, '2021-07-01', 'Bees appear healthy and active'),
    (2, 2, '2021-07-02', 'Queen spotted, laying eggs'),
    (3, 3, '2021-07-10', 'Hive population growing steadily'),
    (4, 3, '2021-07-15', 'Some signs of possible mite infestation'),
    (5, 2, '2021-07-20', 'Honey production looks promising');

    -- Insert test data for MaintenancePlan
    INSERT INTO MaintenancePlan (hive_id, planned_date, description) VALUES
    (1, '2021-10-01', 'Winter preparation'),
    (2, '2021-10-02', 'Winter preparation'),
    (3, '2021-10-05', 'Winter preparation'),
    (4, '2021-10-10', 'Mite treatment'),
    (5, '2021-10-15', 'Winter preparation');

    -- Insert test data for Incident
    INSERT INTO Incident (hive_id, incident_date, description) VALUES
    (4, '2021-07-20', 'Mite infestation detected');

    -- Insert test data for WeatherData
    INSERT INTO WeatherData (region_id, date, temperature, humidity, precipitation) VALUES
    (1, '2021-08-01', 25.5, 60.0, 0.0),
    (2, '2021-08-01', 28.0, 55.0, 0.0),
    (3, '2021-08-01', 23.0, 65.0, 5.0),
    (4, '2021-08-01', 22.0, 70.0, 2.0);

END;
$$ LANGUAGE plpgsql;

-- Execute the function to populate the database
SELECT populate_test_data();

-- Confirm population
SELECT 'Test data inserted successfully!' AS result;

\end{lstlisting}

\subsection{Предложить pl/pgsql-функции и процедуры, для выполнения критически важных запросов (которые потребуются при последующей реализации прецедентов).}
\begin{lstlisting}
    -- Create function to create triggers if they don't exist
CREATE OR REPLACE FUNCTION create_trigger_if_not_exists(
    p_trigger_name TEXT,
    p_table_name TEXT,
    p_trigger_timing TEXT,
    p_trigger_event TEXT,
    p_trigger_function TEXT
) RETURNS VOID AS $$
BEGIN
    -- Check if the trigger already exists
    IF NOT EXISTS (
        SELECT 1
        FROM information_schema.triggers
        WHERE trigger_name = p_trigger_name
          AND event_object_table = p_table_name
    ) THEN
        -- Create the trigger if it doesn't exist
        EXECUTE format(
            'CREATE TRIGGER %I
             %s %s ON %I
             FOR EACH ROW
             EXECUTE FUNCTION %s()',
            p_trigger_name,
            p_trigger_timing,
            p_trigger_event,
            p_table_name,
            p_trigger_function
        );
        RAISE NOTICE 'Trigger % created on table %', p_trigger_name, p_table_name;
    ELSE
        RAISE NOTICE 'Trigger % already exists on table %', p_trigger_name, p_table_name;
    END IF;
END;
$$ LANGUAGE plpgsql;

-- Create triggers
CREATE OR REPLACE FUNCTION update_last_login()
RETURNS TRIGGER AS $$
BEGIN
    NEW.last_login = CURRENT_TIMESTAMP;
    RETURN NEW;
END;
$$ LANGUAGE plpgsql;

SELECT create_trigger_if_not_exists(
    'update_user_last_login',
    'user',
    'BEFORE',
    'UPDATE',
    'update_last_login'
);

-- Create a function to grant access to all regions for admin users
CREATE OR REPLACE FUNCTION grant_admin_access()
RETURNS TRIGGER AS $$
BEGIN
    IF NEW.role = 'ADMIN' THEN
        INSERT INTO "allowed_region" (user_id, region_id)
        SELECT NEW.user_id, r.region_id
        FROM "region" r
        ON CONFLICT ("user_id", "region_id") DO NOTHING;
    END IF;
    RETURN NEW;
END;
$$ LANGUAGE plpgsql;

SELECT create_trigger_if_not_exists(
    'admin_access_trigger',
    'user',
    'AFTER',
    'INSERT OR UPDATE',
    'grant_admin_access'
);

CREATE OR REPLACE FUNCTION update_population_estimate()
RETURNS TRIGGER AS $$
BEGIN
    IF TG_OP = 'INSERT' THEN
        UPDATE "bee_community"
        SET population_estimate = population_estimate + NEW.quantity
        WHERE community_id = (SELECT community_id FROM bee_community WHERE hive_id = NEW.hive_id);
    ELSIF TG_OP = 'DELETE' THEN
        UPDATE "bee_community"
        SET population_estimate = population_estimate - OLD.quantity
        WHERE community_id = (SELECT community_id FROM bee_community WHERE hive_id = OLD.hive_id);
    END IF;
    RETURN NULL;
END;
$$ LANGUAGE plpgsql;

SELECT create_trigger_if_not_exists(
    'update_bee_population',
    'honey_harvest',
    'AFTER',
    'INSERT OR DELETE',
    'update_population_estimate'
);

-- Trigger to automatically create a veterinary passport for new bee communities
CREATE OR REPLACE FUNCTION create_veterinary_passport()
RETURNS TRIGGER AS $$
BEGIN
    INSERT INTO "veterinary_passport" (bee_community_id, issue_date, last_inspection_date)
    VALUES (NEW.community_id, CURRENT_DATE, CURRENT_DATE);
    RETURN NEW;
END;
$$ LANGUAGE plpgsql;

SELECT create_trigger_if_not_exists(
    'new_bee_community_passport',
    'bee_community',
    'AFTER',
    'INSERT',
    'create_veterinary_passport'
);

-- Trigger to update production report when new honey harvest is added
CREATE OR REPLACE FUNCTION update_production_report()
RETURNS TRIGGER AS $$
BEGIN
    INSERT INTO "production_report" (apiary_id, start_date, end_date, total_honey_produced)
    VALUES (
        (SELECT apiary_id FROM "hive" WHERE hive_id = NEW.hive_id),
        DATE_TRUNC('month', NEW.harvest_date),
        DATE_TRUNC('month', NEW.harvest_date) + INTERVAL '1 month' - INTERVAL '1 day',
        NEW.quantity
    )
    ON CONFLICT (apiary_id, start_date, end_date)
    DO UPDATE SET total_honey_produced = "production_report".total_honey_produced + NEW.quantity;
    RETURN NEW;
END;
$$ LANGUAGE plpgsql;

SELECT create_trigger_if_not_exists(
    'update_honey_production',
    'honey_harvest',
    'AFTER',
    'INSERT',
    'update_production_report'
);


-- 1. Функция для получения общего количества меда, собранного в конкретном улье за определенный период
CREATE OR REPLACE FUNCTION get_total_honey_harvested(
    p_hive_id INTEGER,
    p_start_date DATE,
    p_end_date DATE
) RETURNS DECIMAL AS $$
DECLARE
    total_honey DECIMAL;
BEGIN
    SELECT COALESCE(SUM(quantity::DECIMAL), 0)
    INTO total_honey
    FROM honey_harvest
    WHERE hive_id = p_hive_id
    AND harvest_date BETWEEN p_start_date AND p_end_date;

    RETURN total_honey;
END;
$$ LANGUAGE plpgsql;

-- 2. Процедура для добавления нового наблюдения в журнал
CREATE OR REPLACE PROCEDURE add_observation(
    p_hive_id INTEGER,
    p_observation_date DATE,
    p_description TEXT,
    p_recommendations TEXT
) AS $$
BEGIN
    INSERT INTO observation_log (hive_id, observation_date, description, recommendations)
    VALUES (p_hive_id, p_observation_date, p_description, p_recommendations);
END;
$$ LANGUAGE plpgsql;

-- 3. Функция для получения текущего состояния здоровья пчелиной общины
CREATE OR REPLACE FUNCTION get_community_health_status(p_community_id INTEGER)
RETURNS VARCHAR AS $$
DECLARE
    health_status VARCHAR;
BEGIN
    SELECT vp.health_status
    INTO health_status
    FROM veterinary_passport vp
    WHERE vp.bee_community_id = p_community_id
    ORDER BY vp.last_inspection_date DESC
    LIMIT 1;

    RETURN COALESCE(health_status, 'Unknown');
END;
$$ LANGUAGE plpgsql;

-- 4. Процедура для обновления статуса улья
CREATE OR REPLACE PROCEDURE update_hive_status(
    p_hive_id INTEGER,
    p_new_status VARCHAR
) AS $$
BEGIN
    UPDATE hive
    SET current_status = p_new_status
    WHERE hive_id = p_hive_id;
END;
$$ LANGUAGE plpgsql;

-- 5. Функция для расчета средней температуры в регионе за последние N дней
CREATE OR REPLACE FUNCTION get_avg_temperature(
    p_region_id INTEGER,
    p_days INTEGER
) RETURNS DECIMAL AS $$
DECLARE
    avg_temp DECIMAL;
BEGIN
    SELECT AVG(temperature::DECIMAL)
    INTO avg_temp
    FROM weather_data
    WHERE region_id = p_region_id
    AND date >= CURRENT_DATE - p_days;

    RETURN COALESCE(avg_temp, 0);
END;
$$ LANGUAGE plpgsql;

-- 6. Процедура для назначения работника на план обслуживания
CREATE OR REPLACE PROCEDURE assign_maintenance_plan(
    p_plan_id INTEGER,
    p_user_id INTEGER
) AS $$
BEGIN
    UPDATE maintenance_plan
    SET assigned_to = p_user_id
    WHERE plan_id = p_plan_id;
END;
$$ LANGUAGE plpgsql;

-- 7. Функция для проверки доступа пользователя к региону
CREATE OR REPLACE FUNCTION has_region_access(
    p_user_id INTEGER,
    p_region_id INTEGER
) RETURNS BOOLEAN AS $$
DECLARE
    has_access BOOLEAN;
BEGIN
    SELECT EXISTS (
        SELECT 1
        FROM allowed_region
        WHERE user_id = p_user_id AND region_id = p_region_id
    ) INTO has_access;

    RETURN has_access;
END;
$$ LANGUAGE plpgsql;

-- 8. Процедура для регистрации инцидента
CREATE OR REPLACE PROCEDURE register_incident(
    p_hive_id INTEGER,
    p_incident_date DATE,
    p_description TEXT,
    p_severity VARCHAR
) AS $$
BEGIN
    INSERT INTO incident (hive_id, incident_date, description, severity)
    VALUES (p_hive_id, p_incident_date, p_description, p_severity);
END;
$$ LANGUAGE plpgsql;

-- 9. Функция для получения последних показаний датчика
CREATE OR REPLACE FUNCTION get_latest_sensor_reading(p_hive_id INTEGER, p_sensor_type VARCHAR)
RETURNS TABLE (value BYTEA, reading_timestamp TIMESTAMP) AS $$
BEGIN
    RETURN QUERY
    SELECT sr.value, sr.timestamp
    FROM sensor s
    JOIN sensor_reading sr ON s.sensor_id = sr.sensor_id
    WHERE s.hive_id = p_hive_id AND s.sensor_type = p_sensor_type
    ORDER BY sr.timestamp DESC
    LIMIT 1;
END;
$$ LANGUAGE plpgsql;

-- 10. Процедура для создания отчета о производстве
CREATE OR REPLACE PROCEDURE create_production_report(
    p_apiary_id INTEGER,
    p_start_date DATE,
    p_end_date DATE
) AS $$
DECLARE
    total_honey DECIMAL;
    total_expenses DECIMAL;
BEGIN
    -- Расчет общего количества собранного меда
    SELECT COALESCE(SUM(quantity::DECIMAL), 0)
    INTO total_honey
    FROM honey_harvest hh
    JOIN hive h ON hh.hive_id = h.hive_id
    WHERE h.apiary_id = p_apiary_id
    AND hh.harvest_date BETWEEN p_start_date AND p_end_date;

    -- Здесь должен быть расчет общих расходов (пример)
    total_expenses := 1000.00;

    -- Создание отчета
    INSERT INTO production_report (apiary_id, start_date, end_date, total_honey_produced, total_expenses)
    VALUES (p_apiary_id, p_start_date, p_end_date, total_honey, total_expenses);
END;
$$ LANGUAGE plpgsql;

\end{lstlisting}

\subsection{Создать индексы на основе анализа использования базы данных в контексте описанных на первом этапе прецедентов. Обосновать полезность созданных индексов для реализации представленных на первом этапе бизнес-процессов.}

\begin{lstlisting}
    -- Add indexes for performance
CREATE INDEX IF NOT EXISTS idx_apiary_manager ON "apiary"(manager_id);

CREATE INDEX IF NOT EXISTS idx_hive_apiary ON "hive"(apiary_id);

CREATE INDEX IF NOT EXISTS idx_bee_community_hive ON "bee_community"(hive_id);

CREATE INDEX IF NOT EXISTS idx_veterinary_passport_community ON "veterinary_passport"(bee_community_id);

CREATE INDEX IF NOT EXISTS idx_veterinary_record_passport ON "veterinary_record"(passport_id);

CREATE INDEX IF NOT EXISTS idx_sensor_hive ON "sensor"(hive_id);

CREATE INDEX IF NOT EXISTS idx_sensor_reading_sensor ON "sensor_reading"(sensor_id);

CREATE INDEX IF NOT EXISTS idx_honey_harvest_hive ON "honey_harvest"(hive_id);

CREATE INDEX IF NOT EXISTS idx_observation_log_hive ON "observation_log"(hive_id);

CREATE INDEX IF NOT EXISTS idx_maintenance_plan_apiary ON "maintenance_plan"(apiary_id);

CREATE INDEX IF NOT EXISTS idx_incident_hive ON "incident"(hive_id);

CREATE INDEX IF NOT EXISTS idx_production_report_apiary ON "production_report"(apiary_id);

CREATE INDEX IF NOT EXISTS idx_weather_data_region ON "weather_data"(region_id);

CREATE UNIQUE INDEX IF NOT EXISTS idx_allowed_region_user_region
ON "allowed_region" (user_id, region_id);

CREATE INDEX IF NOT EXISTS idx_allowed_region_region ON "allowed_region"(region_id);

CREATE INDEX IF NOT EXISTS idx_region_apiary_apiary ON "region_apiary"(apiary_id);

CREATE INDEX IF NOT EXISTS idx_region_apiary_region ON "region_apiary"(region_id);

\end{lstlisting}

\end{document}
